\chapter{Motivación}
\minitoc
\label{chap:motivacion}

La Inteligencia Artificial estudia como crear sistemas que se comporten de manera inteligente, es decir, que sean capaces de razonar, deducir y resolver problemas del mismo modo que pudiera hacerlo una persona. Se busca que estos sistemas sean autónomos, se busca que estos sistemas sepan justificar sus resultados, se busca que estos sistemas sean capaces de aprender para poder desempeñar su tarea con mejores resultados o en menos tiempo. Pero la inteligencia conlleva más cosas, tales como la creatividad. La creatividad es el impulso por el cual alguien (o algo) decide crear una obra de la nada. No solo una obra artística, también obras funcionales como lo fueron los grandes inventos del pasado. 

Existe, no obstante, controversia con respecto al campo, como suele suceder siempre que un ordenador empieza a conseguir hacer lo que antes solo podían hacer las personas. En el caso del sistema Emily Howell, por ejemplo, ha habido numerosos directores de orquestas que se han negado a interpretar sus composiciones al no provenir de un compositor humano. Existe, a ojos de los más conservadores respecto al tema, el miedo a que el esfuerzo de la composición musical pierda su significado. Si no podemos distinguir además qué piezas han sido compuestas por máquinas y cuales no, el problema se acentúa. 

En este caso, la motivación del trabajo es una mezcla entre creatividad y la necesidad de solucionar un problema. Se habla de creatividad porque el proyecto está aplicado a un campo inherentemente creativo, como es la música. Pero al mismo tiempo, pretende ser una herramienta que ayude a estudiantes de música a progresar en su trabajo. Este sistema inteligente, será capaz de razonar, deducir, y último lugar crear, la armonía de piezas musicales sencillas. Si bien esto no es un trabajo completo de composición, sí que debería ayudar a corregir partituras allí donde el sistema detecte incoherencias con la armonía creada o ya presente en la pieza. 


El interés por la aplicación del \textit{Answer Set Programming} en la música derivó de un trabajo previo con el Profesor Cabalar durante el curso de la asignatura de Representación del Conocimiento y Razonamiento Automático donde los alumnos construimos un sistema que dada una melodía, producía un canon polifónico. El presente trabajo se planteó como una extensión generalista del problema, orientado a composición completa, aunque fue reducido para poder entrar en los límites de un trabajo de fin de grado debido a la gran complejidad de crear un sistema así. No obstante, la lógica proposicional es idónea para esta tarea ya que el conjunto de reglas de la armonía clásica usada en los niveles más elementales del Conservatorio no ha cambiado desde los orígenes de la materia. Es un conjunto de reglas conciso, no muy grande y más o menos estricto. Simplemente traduciendo este conjunto de reglas a restricciones del lenguaje de lógica proposicional y siendo capaces de extraer los hechos lógicos de una partitura, el sistema debería ser capaz de detectar los errores de la misma y solucionarlos, así como rellenar huecos dejados a propósito en la partitura o completar otras líneas melódicas para formar, por fin, la armonía de la canción.

Con fines académicos, el sistema podrá ayudar a estudiantes y profesores de armonía básica por igual, al poder corregir ejercicios de esta materia de modo rápido, incluso proponiendo soluciones no contempladas inicialmente. También ayudará en trabajos de composición armónica, explorando, de forma creativa, nuevas posibilidades para el autor de la pieza original. No solo es una propuesta interesante a nivel de investigación informática, sino también a nivel musical. Por último, pero no menos importante, cada vez más alumnos cursan estudios musicales de algún tipo desde edades más tempranas, ya sea por cuenta propia o a través de academias y conservatorios. El estudio del contexto ha revelado que si bien existe investigación con respecto a la computación musical, no hay muchos sistemas inteligentes finales dedicados a la composición o a la asistencia en la composición musical.